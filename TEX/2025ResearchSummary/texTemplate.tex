%%%%%%%%% JPS abstract %%%%%%%%%%%%%%%%%%%%%%%%%%%%%%%%%%%%%%%%%%
\documentclass[12pt,a4paper,upLaTeX]{jsarticle}

%%%%%%%%% packages %%%%%%%%%%%%%%%%%%%%%%%%%%%%%%%%%%%%%%%%%%%%%%
\usepackage{graphicx} % Include figure files
\usepackage[%                           % 余白の設定
%mag=1400,%                              jarticle の場合(14ptに)
dvipdfm,truedimen,%
top=30truemm,bottom=20truemm,%
left=20truemm,right=20truemm]{geometry}
\pagestyle{empty}

%%%%%%%%% header %%%%%%%%%%%%%%%%%%%%%%%%%%%%%%%%%%%%%%%%%%%%%%%%
\begin{document}
\vspace{-5pt}
\begin{center}
{\gt \Large J-PARC チャームバリオン分光実験のためのリアルタイム事象識別プロセスの開発と評価 }\\[14pt]

\end{center}

\vspace{10pt}
%%%%%%%%%%%%%%%%%%%%%%%%%%%%%%%%%%%%%%%%%%%%%%%%%%

\subsubsection*{研究の背景と目的}

この世に存在する力は4種類の相互作用で記述されることがわかっている。その一種である「強い相互作用」でクォークが結合した粒子はハドロンと呼ばれる。強い相互作用を記述する理論である量子色力学(QCD)の基本方程式自体は知られており、高エネルギー実験で精密に検証されている。その一方、ハドロンが存在できるような低エネルギー領域ではクォーク同士の結びつきが非常に強くなるため、この方程式を解くのが困難になる。そのため、本来は軽い粒子であるクォークから100倍も重い陽子や中性子が形成されるメカニズムや、ハドロン内部に閉じ込められたクォークの振る舞いといった、特に低エネルギー領域でQCDが紡ぐ現象の解明が課題となっている。

クォーク3つで構成されるハドロンはバリオンと呼ばれる。その中でもチャームクォークを1つ含むチャームバリオンはバリオンの内部に存在すると考えられているクォーク対の運動の相関であるダイクォーク相関を調べるのに適している。特にチャームバリオンの励起状態の生成率や崩壊分岐比を測定することでダイクォーク相関といった、バリオン内部に閉じ込められたクォークの運動の性質を調べることができる。

J-PARC 高運動量二次粒子ビームラインにおいてチャームバリオンの分光実験を計画されている。この実験では20 $\mathrm{GeV}/c$の$\pi^-$中間子ビームを液体水素標的に衝突させ、チャームバリオンの励起状態$Y_c^{*+}$と$D^
{*-}$中間子を生成する。この$\pi^-+p\rightarrow Y_c^{*+}+D^{*-}$反応において、ビーム粒子運動量と散乱$D^{*-}$の終状態粒子である$K^+,\,\,\pi^-,\,\,\pi^-$の運動量から欠損質量法によってチャームバリオンの質量スペクトルを測定し、それに現れるチャームバリオンの基底状態の生成率や励起状態の生成・崩壊分岐比からバリオン内部のクォーク対の運動であるダイクォーク相関を調べることを目的としている。

この実験で想定される高い信号レートの環境下でチャームバリオン生成事象を効率的に収集するには複雑な事象選別が要求されるため、我々はハードウェアトリガーを撤廃した連続読み出し式データ収集システムとソフトウェアによるオンラインフィルタープロセスを組み合わせたデータ取得システム(NestDAQ)を採用する。

私はこの連続読み出し式データ収集システムにおいて計算機ネットワークへ送られる無選別のデータから不要な事象を除去するためのオンラインフィルタープロセスの開発を行っている。この研究のテーマは開発したオンラインフィルタープロセスのデータレート削減性能の評価とリアルタイムの処理に必要な計算資源の調査である。

\subsubsection*{研究計画と研究成果}

2025年度と2026年度にてビーム粒子が標的と反応せずに通過する事象を削減するオンラインフィルタープロセスの開発と評価を行うことを計画している。

2025年度は検出器間の粒子の飛行時間の算出や多数のチャンネルを持つ検出器から送られてくるヒットデータを検出器名と対応させるアルゴリズムといったオンラインフィルターの基礎となる要素技術の開発・評価を行なった。日本物理学会2025年秋季大会や計測システム研究会で研究成果の口頭発表を行なった。また、日本物理学会2026年春季大会においても口頭発表を行う予定である。

2026年度は2025年度に開発した要素技術を用いて、ビーム粒子が標的と反応せずに通過する事象を選別するオンラインフィルタープロセスの実装を行う。過去のビームタイム$^{[1]}$で取得されたデータを用いて、事象選別によるデータレート削減率の評価と必要な計算資源の調査を行い、その結果を修士論文にまとめることを計画している。


[1] K. Shirotori et al., J-PARC proposal T103, “Proposal for a test experiment to evaluate the
performance of the trigger-less data-streaming type data acquisition system” (2024)



\end{document}