\documentclass[../../main.tex]{subfiles}
\usepackage{amsmath,amsthm,amsfonts,latexsym,mathtools,bm,ulem,amssymb,tikz,circuitikz,graphicx}
\usepackage{times}
\usepackage[subrefformat=parens]{subcaption}
\usepackage{here}
\usepackage{siunitx}
\usepackage{physics}
\usepackage{mhchem}
\usepackage{upgreek}%ギリシャ文字を立てるのだ
\numberwithin{equation}{section}
\numberwithin{table}{section}
\numberwithin{figure}{section}
\usepackage{wrapfig}%文中に画像を入れる
\begin{document}

\section{解析手法}
DRS4で取得したデータの解析手法について述べる。後述の解析に用いるためにトリガー時間、ADCSumの2つを求めた。トリガー時間はcoincidenceの判定、ADCSumは$\upgamma$線がカウンタで落としたエネルギーを求めるのに用いる。

\subsection{トリガー時間}\label{sec:trigger_time}
取得された波形データにはセルごとの電圧値が記録されているが、どのセルで実際にDAQしきい値を越えたのかは記録されていない。後の解析で使うためにパルス信号が何セル目でトリガーしきい値を越えたのかを取得する。図\ref{fig:waveform_example_discriLine}の例では、赤線(DAQで想定したしきい値)と青線(波形)の交わるセルが何番目か取得する。

				\begin{figure}[H]
					\centering
					\includegraphics[height=6cm]{../FIGURE/exp5/waveform_example_discriLine.pdf}
					\caption{PMTの波形の例。横軸はトリガーセルから数えたセル番号、縦軸はそのセルに記録された電圧値である。赤い線はDAQのトリガーしきい値$-20$ \si{\milli\volt}であり、赤い線と青い線が交わるときのセルを求めた。}
					\label{fig:waveform_example_discriLine}
				\end{figure}

        波形をトリガーセルのはじめから順番に参照していくとき、初めて3セル連続でトリガーしきい値を越えた場合に、その3つの連続したセルのうち最初のセルをトリガーセルとした。これはPMTの出力パルスに存在するジッタを考慮し、信頼性を上げるためである。
        予備実験では1 Gsps、本実験の物理測定では0.7 Gspsでデータ取得を行った。求めたトリガーセルからサンプリングレートを用いて時間に変換し、各イベントごとのトリガー時間を求めた。

\subsection{ADCSum}
図\ref{fig:NaIwaveform}, \ref{fig:GSOwaveform}のように、用いるシンチレータによって波形が異なる。減衰時間はNaIシンチレータは250 ns程度、GSOシンチレータは30 -- 60 ns程度である。

データ取得始めの時間0 nsから、終わりの時間の40分の1すなわち30 / 36.6 ns(予備実験/物理測定)の範囲で電圧の値を足し、足した数で割ることでpedestalを求めた。減衰時間の違いから、NaIシンチレーションカウンタではトリガー時間を0 nsとして[-50,650] nsの700 nsの範囲、GSOシンチレーションカウンタではトリガー時間から[-50, 180] nsの230 nsの範囲で、各セルごとの電圧値からpedestalの値を引いた値の和をADCSumとした。

%1 Gspsだと0 -- 1024 ns
%0.7 Gspsだと0 -- 1463 ns

%fPedestalTmin = fTime[0][0][0];
%fPedestalTmax = fTime[0][0][1023] / 40.0;

\begin{figure}[H]
  \begin{minipage}{0.5\hsize}
    \centering
    \includegraphics[height=5cm]{../FIGURE/analysis/NaIwaveform.png}
    \caption{NaI(Tl)シンチレーションカウンタによる波形の例}
    \label{fig:NaIwaveform}
  \end{minipage}
  \begin{minipage}{0.5\hsize}
    \centering
    \includegraphics[height=5cm]{../FIGURE/analysis/GSOwaveform.png}
    \caption{GSO(Ce)シンチレーションカウンタによる波形の例}
    \label{fig:GSOwaveform}
  \end{minipage}
\end{figure}
%これずれるのなんとかならんかな?w

ADCSumとエネルギーの対応については、付録\ref{exp1}で述べる。



\end{document}