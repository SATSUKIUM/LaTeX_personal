\documentclass[../../main.tex]{subfiles}
\usepackage{amsmath,amsthm,amsfonts,latexsym,mathtools,bm,ulem,amssymb,tikz,circuitikz,graphicx}
\usepackage{times}
\usepackage[subrefformat=parens]{subcaption}
\usepackage{here}
\usepackage{siunitx}
\usepackage{physics}
\usepackage{mhchem}
\usepackage{upgreek}%ギリシャ文字を立てるのだ
\numberwithin{equation}{section}
\numberwithin{table}{section}
\numberwithin{figure}{section}
\usepackage{wrapfig}%文中に画像を入れる
\begin{document}
	\FloatBarrier
	\section{全体のまとめ}
	本実験ではBellの不等式の破れを証明するための実験を行った。先行研究より大きなシンチレータ、多くのシンチレーションカウンタを用いて統計量を貯める工夫をした。得られた結果はフィッティングが上手くいかず、$\kappa=1.09032\pm 0.11722$と$\kappa$が1を超える結果となった。原因はカウンタごとの個体差だと考えられる。そこで、カウンタごとの収率を求め、データを補正した。その結果$\kappa=0.8898\pm 0.120$と1$\sigma$の精度でBellの不等式が破れていることを示せた。
	% conclusion-future
\end{document}