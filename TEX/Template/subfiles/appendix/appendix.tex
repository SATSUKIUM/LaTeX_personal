\documentclass[../../main.tex]{subfiles}
\usepackage{amsmath,amsthm,amsfonts,latexsym,mathtools,bm,ulem,amssymb,tikz,circuitikz,graphicx}
\usepackage{times}
\usepackage[subrefformat=parens]{subcaption}
\usepackage{here}
\usepackage{siunitx}
\usepackage{physics}
\usepackage{mhchem}
\usepackage{upgreek}%ギリシャ文字を立てるのだ
\numberwithin{equation}{section}
\numberwithin{table}{section}
\numberwithin{figure}{section}
\usepackage{wrapfig}%文中に画像を入れる
\begin{document}
\FloatBarrier
\section{その他}

\subsection{本実験の測定期間と環境モニタリング}\label{sec:exp5_run_period}
				大阪大学理学部棟H009実験室にて$20$ \si{\degreeCelsius}の暖房をかけながら\footnote{この時期の外気温は高くとも10度に満たず、暖房の$20$ \si{degreeCelsius}というのはつまるところ実験室の温度を一定に保ったことにほかならない。}\footnote{データロガーでとってた温度変化の様子も載せられたら良いかも。特に温度変化が$\pm1$ \si{\degreeCelsius}以内に収まってたことを主張するために。}、2025年1月--日から2025年2月--日の間の合計約304時間、データ取得を行った。データ取得中のDAQレートは概ね6 \si{\hertz}であり、総取得データ数は6,206,964であった。
				以下の図\ref{fig:temp}は実験室の、特にPMTの周囲の温度変動のグラフである。また、表\ref{table:temp}には温度変動の上下端と変動範囲を示した。これによれば、実験室の温度変動は3 \si{\degreeCelsius}以内に収まっていることがわかる。なお$\mathrm{A1}$のグラフが2025年1月1日14:00ほどを境に大きく変動しているが、他のセンサーの示す値が大きく変動していないことと地下にある実験室の温度が短時間に急変することは考えにくいため、センサーの電気的な接触異常等によるものと考えている。

				\begin{figure}[b]
					\centering
					\includegraphics[width=0.9\columnwidth]{../FIGURE/exp5/temp.pdf}
					\caption{PhysicsRunにおける実験室の温度変動}
					\label{fig:temp}
				\end{figure}


				\begin{table}[b]
					\centering
					\caption{実験室の温度変動の上下端と変動範囲}\label{table:temp}
					\begin{tabular}{c|cccc}
						& A1 & A2$_{0^{\circ}}$ & S2 & S1 \\ \hline
						最大温度 (\si{\degreeCelsius}) & 21.3 & 22.9 & 20.9 & 21.3 \\
						最低温度 (\si{\degreeCelsius}) & 18.4 & 19.9 & 18.6 & 19.5 \\
						温度変化の範囲 (\si{\degreeCelsius}) & 2.9 & 3.0 & 2.4 & 1.8 \\
					\end{tabular}
				\end{table}

        \FloatBarrier
\subsection{パルス波形デジタイザー: DRS4 Evaluation Board}
PMTの波形の記録のために用いたDAQシステムであるPSI社のDRS4 Evaluation Boardについて述べる。DRS4 Evaluation Boardは内部に1024個のキャパシタ(それぞれをセルと呼ぶ)と読み出し回路からなるインプットを4つ(4CH)持つパルス波形デジタイザーである。内部のクロックに従って各セルを巡回しつつ、順次それぞれのセルにパルスの振幅に比例した電荷が充電されてゆく。DAQの際には発行されるセルフトリガー(もしくは入力される外部トリガー)があれば各セルへの充電はストップされ、巡回してるセルたちのうち、トリガーが発行されたタイミングでのセルから電荷の読み出し(デジタイズ)が行われる。セルの巡回の速さは内部のクロックによって制御されており、サンプリングレートは$0.7$--$5$ GSPsまで変更できる。1024セルあるので記録される波形の時間幅は$0.7$ GSPsのときは$1024\times\frac{1}{0.7}\sim1462$ nsで、$5$ GSPsのときは$1024\times\frac{1}{5}\sim205$ nsである。

\FloatBarrier
\subsubsection{DAQ制御ソフト: DRSOsc}
  DRS4の制御には同じくPSI社のDRSOscというGUIソフトを用いた。\footnote{GUIの写真載せておこうかな} DRSOscはDRS4からの波形情報をバイナリファイルに順次書き込んで記録することができる。

  DRSOscではDRS4 Evaluation Board内部の$100$ \si{\mega\hertz}のクロックを用いてDRS4の各セルの固有の時間幅(もし$1$ GPSsで動作する場合は1セルあたり約$1$ nsの時間幅となるが、各セルによって少しのばらつきがある)を測定し、波形の時間情報の較正用データを取得できる。DRS4 Evaluation Boardでは波形の読み出しを開始するセルがイベントによってことなるためにこのような較正が必要なのである。

  DRS4 Evaluation Boardから送られてくるバイナリデータには、上記の時間較正データと、各イベントごとにイベント番号、日時(ミリ秒まで)と波形読み出しを始めるセル(トリガーセルと呼ぶ)と各チャンネルごとの電圧値がトリガーセルから順に1024個などが含まれている。(参考: 図\ref{fig:preparation_binary_example})

  \begin{figure}[H]
    \begin{minipage}[b]{0.48\columnwidth}
      \centering
      \includegraphics[width=\columnwidth]{../FIGURE/preparation/binary_example01.png}
      \subcaption{バイナリファイルの構成(1)}\label{fig:preparation_binary01}
    \end{minipage}
    \hspace{0.04\columnwidth} % ここで隙間作成
    \begin{minipage}[b]{0.48\columnwidth}
      \centering
      \includegraphics[width=\columnwidth]{../FIGURE/preparation/binary_example02.png}
      \subcaption{バイナリファイルの構成(2)}\label{fig:preparation_binary02}
    \end{minipage}
    \caption{DRSOscにて取得できる波形のバイナリファイル\cite{preparation:DRS4_manual}}\label{fig:preparation_binary_example}
  \end{figure}

\FloatBarrier
\subsubsection{DRSOscでのDAQロジック}
  DRS4 Evaluation Boardでは4つのインプットの信号のANDやORでデータ取得のトリガーロジックを組むことができ、実験5: PhysicsRunでは散乱体$S1$, $S2$と吸収体$A1$のANDロジックでデータ取得を行った。

  DRS4 Evaluation Boardには複数台のDRS4 Evaluation Boardを使用して4チャンネル以上のインプットを持つDAQシステムを構築する機能がある。複数台のDRS4 E.B.は一台をマスターボードとして波形読み出しのトリガー信号を伝播させていく\footnote{トリガー信号の伝播には1台あたり概ね$16$ nsの遅れが発生する。}デイジーチェインを構成することができる。(これはdaisy-chain modeと呼ばれる。) 我々は2台のDRS4 E.B.をdaisy-chain modeで動作させた。

\FloatBarrier
\subsection{GSO結晶のPMTへの固定}
我々は実験5(Bellの不等式の破れの検証実験)にて広い範囲で散乱体からの$\upgamma$線を捉えるために大きいサイズのGSO(Ce $5$\%)無機シンチレータを用いた。
% 「本実験」ってどれよ?って思ったので明記した。あと、GSOもほかと比較の上選んだわけではないから用いたという言葉にとどめた。
$2.0\times2.0\times12.0$ \si{\centi\meter^3}の四角柱の形状のものを用いた。(図\ref{fig:preparation_GSO_crystal})

GSO結晶の密度は$6.7$ \si[per-mode=symbol]{\gram/\cubic\centi\meter}と高いので、図(\ref{fig:preparation_PMT_and_crystal})のようにPMTと結晶を水平に接続しようとするとき、GSO結晶が重力で落ちてしまわないように固定することが必要である。そこで我々は図\ref{fig:preparation_PMT_crystal_resin}に透明の円形のパーツで示したような固定具を作成した。設計は3DCADソフトAutodesk Fusion360を使い、印刷は光造形3DプリンタElegoo Saturn2を用いた。
\begin{figure}[H]
  \centering
  \includegraphics[height=5cm]{../FIGURE/GSO_crystal.jpg}
  \caption{本実験に用いたGSO(Ce)シンチレータ}\label{fig:preparation_GSO_crystal}
\end{figure}

\begin{figure}[H]
  \begin{minipage}[b]{0.48\columnwidth}
    \centering
    \includegraphics[width=\columnwidth]{../FIGURE/PMT_and_crystal.jpg}
    \subcaption{PMTとシンチレータのサイズの外観}\label{fig:preparation_PMT_and_crystal}
  \end{minipage}
  \hspace{0.04\columnwidth} % ここで隙間作成
  \begin{minipage}[b]{0.48\columnwidth}
    \centering
    \includegraphics[width=\columnwidth]{../FIGURE/PMT_crystal_resin.jpg}
    \subcaption{PMTとシンチレータの固定のためのパーツ}\label{fig:preparation_PMT_crystal_resin}
  \end{minipage}
  \caption{GSOシンチレータ固定具の概要}
\end{figure}
\subsubsection{固定具の寸法}
  断面サイズ$2\times2$ \si{\square\centi\meter}のGSO結晶に対して集光用のテープなどを巻いてから固定することを見据えて、以下のようにシンチレータの太さより、テープの厚みだけ余裕を持たせて設計を行った。

  図\ref{fig:preparation_dimension1}のスケッチの形状を$Y$軸回転させ、その上面を図\ref{fig:preparation_dimension2}のスケッチのとおりに原点が中心の正方形でくり抜いた。その後適度に面取りをして図\ref{fig:preparation_dimension3}のとおりに設計をした。

  これを図\ref{fig:preparation_PMT_crystal_resin}の配置でPMTに固定する。まずはPMTのアウターチューブとシンチレータ固定具をテープで巻くことで固定する。その次に後述の集光用テープで巻いたGSOシンチレータの片端を固定具に差し込み、GSOシンチレータの他方の端からテープで張るようにしてPMTに固定する。(図\ref{fig:preparation_GSO_fix})
  \begin{figure}[H]
    \begin{minipage}[b]{0.48\columnwidth}
      \centering
      \includegraphics[width=\columnwidth]{../FIGURE/fix_parts_01.png}
      \subcaption{寸法1}\label{fig:preparation_dimension1}
    \end{minipage}
    \hspace{0.04\columnwidth} % ここで隙間作成
    \begin{minipage}[b]{0.48\columnwidth}
      \centering
      \includegraphics[width=\columnwidth]{../FIGURE/fix_parts_02.png}
      \subcaption{寸法2}\label{fig:preparation_dimension2}
    \end{minipage}
    \begin{minipage}[b]{0.48\columnwidth}
      \centering
      \includegraphics[width=\columnwidth]{../FIGURE/fix_parts_03.png}
      \subcaption{寸法3}\label{fig:preparation_dimension3}
    \end{minipage}
    \caption{シンチレータ固定具の寸法と完成図}
  \end{figure}

  \begin{figure}[H]
    \begin{minipage}[b]{0.4\columnwidth}
      \centering
      \includegraphics[width=\columnwidth]{../FIGURE/GSO_fix_01.jpg}
      \subcaption{集光用のテープを巻いたGSO結晶を固定具に差し込んだ様子}\label{fig:preparation_GSO_fix_01}
    \end{minipage}
    \hspace{0.04\columnwidth} % ここで隙間作成
    \begin{minipage}[b]{0.4\columnwidth}
      \centering
      \includegraphics[width=\columnwidth]{../FIGURE/GSO_fix_02.jpg}
      \subcaption{シンチレータ固定具をPMTのアウターチューブにテープで固定した様子}\label{fig:preparation_GSO_fix_02}
    \end{minipage}
    \begin{minipage}[b]{0.4\columnwidth}
      \centering
      \includegraphics[width=\columnwidth]{../FIGURE/GSO_fix_03.jpg}
      \subcaption{GSO結晶の「頭」をテープで張ることでPMTへ固定した様子}\label{fig:preparation_GSO_fix_03}
    \end{minipage}
    \caption{GSOシンチレータ固定具の実際の写真}\label{fig:preparation_GSO_fix}
  \end{figure}

\subsubsection{固定具の印刷}
  用いた3Dプリンター、レジン、露光時間などのセッティングを書く

  図\ref{fig:preparation_dimension3}の3DCADデータを3Dプリンタで印刷可能な形式にするためにスライスソフトChituBox Basicを用いてスライスした。このソフトでは3Dプリンタの機種名を選ぶことで基本的な設定がインポートされるようになっており、今回用いた3DプリンタElegoo Saturn2を選択した。使用するレジン(Elegoo Standard Photopolymer Resin)の種類もここで選択した。

  ChituBox Basicではスライスする3Dモデルに自動でサポート用の「足」をつけることが可能であるが、印刷時にその足が本体と融合してしまう不具合が出たため足はつけないことにした。最終的に印刷に成功した印刷パラメータを表\ref{table:preparation_3Dprinter}に記す。
  \begin{table}[H]
    \begin{center}
      \caption{3Dプリンタの印刷パラメータ}\label{table:preparation_3Dprinter}
     \begin{tabular}{l|r}
       項目 & 値 \\\hline\hline
       Layer Hight & 0.050 \si{\milli\meter}  \\
       Bottom Layer Count & 10 \\
       Exposure Time & 25.000 \si{\second} \\
       Bottom Exposure Time & 35.000 \si{\second}
     \end{tabular}
    \end{center}
   \end{table}
\subsubsection{集光用のアルミナイズドマイラー}
  なるべく多くの光をPMTに送るために、GSO結晶の周囲にはシンチレーション光を反射させる目的のアルミナイズドマイラーフィルムを巻いた。図\ref{fig:preparation_alum_mylar}のような展開図の通りにアルミナイズドマイラーフィルムを切り取り、結晶に巻き付けたのちにテープで周囲を巻くことでフィルムを結晶に密着させた。
  \begin{figure}[H]
    \centering
    \includegraphics[width=0.5\columnwidth]{../FIGURE/alum_mylar_01.jpg}
    \caption{アルミナイズドマイラーフィルムの展開図の概要}\label{fig:preparation_alum_mylar}
  \end{figure}



  \FloatBarrier
  \subsection{DRS4の波形情報バイナリの変換と時間較正アルゴリズムの改良}
    DRSOscでDAQするとき、大量の波形データを省スペースで高速に記録するために、波形はある様式に則ったバイナリファイルとして記録される。その様式の詳細はPSIのDRS4 Evaluation Boardのマニュアル\cite{reparation:DRS4_manual}に記載されている。 そのバイナリファイルから波形を読み出すプログラムの改良をし、読み出しのための計算量を大幅に削減したことを報告する。

    \subsubsection{DRSOscのバイナリの様式}
      最終的に得るものは、デジタイズされたパルスの電圧(振幅)と時間のペアの配列である。しかしながらDRSOscのバイナリの記録形式ではそれらはペアとして記録することはなされていない。
      
      DRS4 Evaluation Boardは1024個の電圧サンプリングセルがパルスを順次デジタイズするが、各セルごとでサンプリング時間幅が少し異なっている。例えば$0.7$ GSPsで動作させた場合は1セルあたり$1/(0.7\times10^9)$ \si{\second}つまり、$1.43$ \si{\nano\second}の時間幅でパルスをサンプリングすることになるが、その時間幅はセルごとに個体差があり(これはマニュアルでは有効ビン幅、effective bin widthと呼ばれている)、例えば実験5: PhysicsRunにてマスターボードとして使用したボードでは図\ref{fig:appendix_effective_bin_width_distribution}のように有効ビン幅がばらついている。この有効ビン幅は時間が経っても変化しないので各Runの全てのイベントに有効であり、全てのイベントで繰り返して時間情報をバイナリに記録しなくとも済むので省スペースになる。
      
      DRS4 Evaluation Boardでパルス波形を記録するとき、バイナリファイルには、各イベントごとに読み出しを始めたセルから順にパルス電圧(振幅)が記録される。問題は、各イベントごとにどのセルから読み出しを始めるか(マニュアル\cite{reparation:DRS4_manual}ではトリガーセル、trigger cellと呼ばれている)が異なっていることである。

      \begin{figure}[tbp]
        \centering
        \includegraphics[width=0.85\columnwidth]{../FIGURE/appendix/effective_bin_width_distribution.png}
        \caption{有効ビン幅のばらつきの例}
        \label{fig:appendix_effective_bin_width_distribution}
      \end{figure}
    
    \subsubsection{波形の時間の計算}
      マニュアル\cite{reparation:DRS4_manual}のノーテーションに則って議論を進める。記録された電圧(振幅)の配列のインデックスを$i$ $(0\leq i \leq 1023)$、そのセルの時刻を$t_{ch}[i]$、あるチャンネル$ch$の各セルごとの有効ビン幅の配列を$dt_{ch}[1024]$、トリガーセルが$tcell$番目だとすると
      \begin{align}
        t_{ch}[i]=\sum_{j=0}^{i-1}dt_{ch}\qty[(j+tcell)\%1024]\label{eq:appendix_t_ch}
      \end{align}
      である。($t_{ch}[0]=0$であるような定義。)

      さて、$tcell$は各イベントごとに異なるために、あるイベントの各セルの時刻を計算する必要があり、その計算回数は1イベントあたり
      \begin{align}
        \text{計算回数}&=\text{ボード数2}\times\text{チャンネル数4}\times(0+1+2+3+4+\cdots +1023)\\
        &=4,190,208
      \end{align}
      つまり、なんと$10^6$回もの配列参照を伴う計算が必要なのである。この操作を全てのイベントについて行う必要があり、例えばたった$1,000,000$イベントが記録されたRunのバイナリをROOTのツリーに変換する際には$10^{12}$回もの配列参照が必要になり、オンサイトの解析効率が大きく損なわれる。

    \subsubsection{計算方法の改良}
      有効ビン幅$dt_{ch}[i]$がRunの全てのイベントに有効であることに着目し、各イベントごとに和を取るのではなく事前に和を計算することで計算量を減らす。式\eqref{eq:appendix_t_ch}の右辺で$tcell$をゼロにしたものを累積有効ビン幅$S_{ch}[i]$とし、式\eqref{eq:appendix_cumulative_effective_bin_width}で定義する。図\ref{fig:appendix_cumulative_bin_width}に青い矢印で示した範囲で有効ビン幅の和をとったものが累積有効ビン幅である。
      \begin{align}
        S_{ch}[i]&=\sum_{j=0}^{i-1}dt_{ch}\qty[j\%1024]\label{eq:appendix_cumulative_effective_bin_width}
      \end{align}
      累積有効ビン幅を用いると$t_{ch}[i]$は以下の式\eqref{eq:appendix_cumulative_effective_bin_width}累積有効ビン幅の差で書ける。図\ref{fig:appendix_cumulative_bin_width}の$i$では$t_{ch}[i]$は単に累積有効ビン幅$S[j]$の差を取れば良いように見えるが、$1023$番目のセルを跨ぐ差の処理でコケるため、場合分けをしている。
      \begin{align}
        t_{ch}[i]=\begin{cases}
          S[tcell+i]-S[tcel] & \quad tcell+i\leq1023\\
          S[1023]+S[tcell+i]-S[tcel] & \quad tcell+i>1023
        \end{cases}\label{eq:appendix_cumulative_effective_bin_width}
      \end{align}

      \begin{figure}[tbp]
        \centering
        \includegraphics[width=0.85\columnwidth]{../FIGURE/appendix/cumulative_bin_width.png}
        \caption{累積有効ビン幅のイメージ。横軸にDRS4のチップ内のセル番号をとっており、1023番目のセルがパルスをデジタイズしたあと0番目のセルへ移る様子を表している。}
        \label{fig:appendix_cumulative_bin_width}
      \end{figure}

      累積有効ビン幅をRunごとに一度だけ$1024$個計算すれば、そのRunに含まれる全イベントの時刻$t_{ch}[i]$の計算には単に累積有効ビン幅の差を$1024$回とるだけで良いため、バイナリを読むための計算量がかなり減り、1イベントごとに
      \begin{align}
        \text{計算回数}&=\text{ボード数2}\times\text{チャンネル数4}\times\text{累積有効ビン幅$S[i]$の差2}\\
        &=16
      \end{align}
      つまりたった16回の配列参照の伴う計算で済むことになる。

    



  \FloatBarrier
  \subsection{解析手法: PMTの信号波形からのトリガータイミングの算出}\label{sec:PMT_discriCell}
    取得された波形データにはトリガーセルから順番に電圧値が記録されているが、どのセルで実際にDAQしきい値を越えたのかは記録されていない。後の解析で使うためにパルス信号がトリガーセルから数えて何セル目でトリガーしきい値を越えたのかを取得する。例えば、図\ref{fig:exp5_waveform_example_discriLine}のような波形でいうところの赤い線(DAQで想定したしきい値)と青い線(波形)の交わるセルが何番目か取得する。
    \begin{figure}[tbp]
      \centering
      \includegraphics[width=0.85\columnwidth]{../FIGURE/exp5/waveform_example_discriLine.pdf}
      \caption{PMTの波形の例。横軸にはトリガーセルから数えたセル番号、縦軸にそのセルに記録された電圧値をとっている。赤い線はDAQのトリガーしきい値$-20$ \si{\milli\volt}であり、赤い線と青い線が交わるときのセルが何番目か知りたい。}
      \label{fig:exp5_waveform_example_discriLine}
    \end{figure}
    
    PMTの出力パルスにはジッタがあるので信頼性を上げるために、波形をトリガーセルから順番に参照していくとき、初めて3連続でトリガーしきい値を越えた場合にその3つの連続したセルのうち、最初のセルにてDAQトリガーが掛かっていたと判断した。(これをトリガータイミングのセルと呼ぶ。)

    このアルゴリズムがしっかりと動いているかの例として、実験5にて取得したデータの一部(これをデータセット\ref{fig:exp5_waveform_example_colz}と呼ぶ。\footnote{内部的には、Run\_005.datです。})からそのトリガータイミングのセルを計算してみる。図\ref{fig:exp5_waveform_example_colz}に793527イベントだけ散乱体$S1$のPMT波形の重ね描きを示した。横軸が150弱のあたりで多くのイベントがしきい値$-20$ \si{\milli\volt}を横切っている傾向が見える。この波形に対して上記のアルゴリズムでトリガータイミングのセルを計算し、その分布を図\ref{fig:exp5_triggercell_example01}に示した。たしかにトリガータイミングのセルから数えて150弱のあたりに鋭いピークがあることが分かる。横軸が0のところにもカウントがあるが、これは波形の記録が始まる前に立ち上がったイベントが全て0として記録されている。そのためとても多く見えるが、実際は波形の読み出しが始まる前にパルスが立ち上がったものの合算にすぎない。

    以下、このアルゴリズムで計算したトリガータイミングのセルを波形の立ち上がりのタイミングとする。
    \begin{figure}[tbp]
      \centering
      \includegraphics[width=0.85\columnwidth]{../FIGURE/exp5/waveform_example_colz.pdf}
      \caption{散乱体$S1$のPMTの波形の例。(2次元ヒストグラム) 横軸にはトリガーセルから数えたセル番号、縦軸にそのセルに記録された電圧値をとっている。色の凡例は密度を表しており、青色の領域より黄色の領域の密度が高い。}
      \label{fig:exp5_waveform_example_colz}
    \end{figure}
    \begin{figure}[tbp]
      \centering
      \includegraphics[width=0.85\columnwidth]{../FIGURE/exp5/triggercell_example01.pdf}
      \caption{図\ref{fig:exp5_waveform_example_colz}のトリガータイミングのセルの分布。横軸は図\ref{fig:exp5_waveform_example_colz}と同じである。}
      \label{fig:exp5_triggercell_example01}
    \end{figure}

    波形の立ち上がりのタイミングを計算できたので、タイムウォークまで見えているか確かめてみる。データセット\ref{fig:exp5_waveform_example_colz}を使って、図\ref{fig:exp5_adc_time_huruno1}に散乱体$S1$のADCと波形立ち上がりのタイミングの分布を示した。
    
    この図を理解するためには次の極端な例を考えなければならない。例えばDAQロジックに入っている散乱体$S1$, $S2$と吸収体$A1$それぞれからDRS4 Evaluation Boardまでの特性インピーダンス$50$ \si{\ohm}のケーブル長さが$0$ m, $10$ m, $20$ mだったとする。この場合、もしも$S1$, $S2$, $A1$から同時に信号が出たとしてもDRS4にインプットされるまでにそれぞれ$50$ nsの差が生じる。この場合、DRS4 Evaluation BoardにおいてDAQトリガーが発行されるのは決まって$A1$の立ち上がるタイミングである。\sout{3つのインプットのうち、一番遅い信号がDAQロジックの最後のピースを埋めるのである。}次に$S1$, $S2$, $A1$からのケーブルがそれぞれ$0$m, $1$ m, $2$ mの例を考える。この場合も$A1$がDAQトリガーを発行するので、記録される波形を見れば$A1$の立ち上がりのタイミングは変わらないが、$S1$, $S2$の立ち上がりは遅く見える。\sout{とにかく一番遅く来たカウンターの信号がDAQトリガーの信号を発行するのである。}記録される各イベントにおいて、$S1$か$S2$か$A1$のどれか1つだけはセルフトリガーでDAQされている。
    
    その認識のうえ、図\ref{fig:exp5_adc_time_huruno1}, 図\ref{fig:exp5_adc_time_huruno2}, 図\ref{fig:exp5_adc_time_sato}を見る。(全てデータセット\ref{fig:exp5_waveform_example_colz}を使用している。) 図\ref{fig:exp5_triggercell_example01}では単一のピークであった波形立ち上がりのタイミングが、ADCとの関係を見ると図\ref{fig:exp5_adc_time01}の右端で2つの系列に分かれていることが分かる。(図\ref{fig:exp5_adc_time02}で赤い線を引いた斜めの系列と、その右で上下方向に伸びる系列) ここでじっくりと図を見ると赤い線の斜めの系列ではパルス立ち上がりのタイミングとADCに関係があり、その右の系列ではそれらに関係がないことが分かる。赤い線の系列ではADCが大きいほどパルス立ち上がりのタイミングが早くシフトしている。これをパルスのタイムウォークであると考えた。そして、右の系列においてはそれより右に分布がないことから、$S1$のセルフトリガーのイベントであると考えた。

    S2を見る。図\ref{fig:exp5_adc_time_huruno2}では系列が1つしか見られないが、上下方向に分布しているために$S2$のセルフトリガーのイベントが多いのだと解釈した。

    A1を見る。図\ref{fig:exp5_adc_time_sato}を見ると、$S1$よりわかりやすく斜めの系列と上下方向に伸びる系列が見える。図\ref{fig:exp5_adc_time04}で赤い線を引いた斜めの系列ではタイムウォークが見えており、その右の系列では$A1$のセルフトリガーであるためにタイムウォークが存在しない。

    \begin{figure}[tbp]
      \begin{minipage}[b]{0.48\columnwidth}
        \centering
        \includegraphics[width=\columnwidth]{../FIGURE/exp5/time_walk/huruno1_wo_title.pdf}
        \subcaption{横軸に波形の電圧値の和、つまりパルスの電荷量に比例した量、縦軸に波形立ち上がりのタイミング(セル)をとっている。}\label{fig:exp5_adc_time01}
      \end{minipage}
      \hspace{0.04\columnwidth} % ここで隙間作成
      \begin{minipage}[b]{0.48\columnwidth}
        \centering
        \includegraphics[width=\columnwidth]{../FIGURE/exp5/time_walk/huruno1_with_title_with_line.pdf}
        \subcaption{図\ref{fig:exp5_adc_time01}に系列の補助線を引いた図}\label{fig:exp5_adc_time02}
      \end{minipage}
      \caption{散乱体$S1$のADCと波形立ち上がりの関係}\label{fig:exp5_adc_time_huruno1}
    \end{figure}

    \begin{figure}[tbp]
      \centering
      \includegraphics[width=0.85\columnwidth]{../FIGURE/exp5/time_walk/huruno2_wo_title.pdf}
      % \subcaption{横軸に波形の電圧値の和、つまりパルスの電荷量に比例した量、縦軸に波形立ち上がりのタイミング(セル)をとっている。}\label{fig:exp5_adc_time01}

      \caption{散乱体$S2$のADCと波形立ち上がりの関係。横軸に波形の電圧値の和、つまりパルスの電荷量に比例した量、縦軸に波形立ち上がりのタイミング(セル)をとっている。}\label{fig:exp5_adc_time_huruno2}
    \end{figure}

    \begin{figure}[tbp]
      \begin{minipage}[b]{0.48\columnwidth}
        \centering
        \includegraphics[width=\columnwidth]{../FIGURE/exp5/time_walk/sato_wo_title.pdf}
        \subcaption{横軸に波形の電圧値の和、つまりパルスの電荷量に比例した量、縦軸に波形立ち上がりのタイミング(セル)をとっている。}\label{fig:exp5_adc_time03}
      \end{minipage}
      \hspace{0.04\columnwidth} % ここで隙間作成
      \begin{minipage}[b]{0.48\columnwidth}
        \centering
        \includegraphics[width=\columnwidth]{../FIGURE/exp5/time_walk/sato_with_title_with_line.pdf}
        \subcaption{図\ref{fig:exp5_adc_time03}に系列の補助線を引いた図}\label{fig:exp5_adc_time04}
      \end{minipage}
      \caption{吸収体$A1$のADCと波形立ち上がりの関係}\label{fig:exp5_adc_time_sato}
    \end{figure}

    ところで、DAQロジックに入っている散乱体$S1$, $S2$と吸収体$A1$からのケーブル長の差は長くとも数cmに収まるようにセットアップを組んだのにも関わらず、$S2$ではセルフトリガーが主な成分であることが図\ref{fig:exp5_adc_time_huruno2}より読み取れる。考えられる要因として、$S1$と$S2$には同じ型番のPMTを用いたが$A1$には違う型番のPMTを用いたことと、$S1$と$S2$のゲインの個体差が挙げられる。$S1$, $S2$にはR878を、$A1$にはR329-02を用いており、内部構造が違っている\footnote{ダイノードの数の違いでダイノード間電圧差で電子の移動速度が違ってるんじゃないかって思ったけど、よくわからなかった。}。また、$S1$と$S2$には同じPMTを用いているが、図\ref{fig:exp5_adc_time01}と図\ref{fig:exp5_adc_time_huruno2}の横方向に伸びた系列のADC値を見るに$S2$より$S1$のほうが少しゲインが高いために、$S1$と$S2$間のタイムウォークによって$S2$のセルフトリガーが主になったのではないかと考えられる。
  \FloatBarrier
  \subsection{失敗集}
    テストランの失敗などをまとめる
    \begin{itemize}
      \item 地下実験室の温度変化
    \end{itemize}
\end{document}